\documentclass{standalone}
\usepackage{mathematico}

\begin{document}
\begin{tikzpicture}[scale=1.25, transform shape]
    \path[use as bounding box] (-2, -1) rectangle (12, 1);

    \draw[<->, >=latex, thick, gray, line width=1.5pt] (-1, 0) -- (11, 0);

    \draw[
      color=mygreen,
      fill=mygreen,
      opacity=0.25
    ] (1, -.5) rectangle (2.5, .5);

    \draw[
      color=mygreen,
      fill=mygreen,
      opacity=0.25
    ] (5, -.5) rectangle (8.5, .5);

    % \draw[thick, gray] (0.5, -.5) rectangle (1.5, .5);
    % \draw[thick, gray] (2, -.5) rectangle (3.5, .5);
    \draw[thick] (6, -.5) rectangle (7.5, .5);
    % \draw[thick, gray] (9, -.5) rectangle (9.5, .5);

    % \draw[thick, dashed, <->, >=latex, gray] (-.5, -.5) -- (.45, -.5) -- (.45, .5) -- (-.5, .5);
    % \draw[thick, dashed, gray] (1.55, -.5) rectangle (1.95, .5);
    \draw[thick, dashed] (3.55, -.5) rectangle (5.95, .5);
    \draw[thick, dashed] (7.55, -.5) rectangle (8.95, .5);
    % \draw[thick, dashed, <->, >=latex, gray] (10.5, -.5) -- (9.55, -.5) -- (9.55, .5) -- (10.5, .5);

    % \node[] at (1, .25) {f1};
    % \node[] at (2.75, .25) {f2};
    \node[] at (6.75, .25) {f3};
    % \node[] at (9.25, .25) {f4};
    % \node[] at (4.75, .25) {identity};

	\end{tikzpicture}
\end{document}